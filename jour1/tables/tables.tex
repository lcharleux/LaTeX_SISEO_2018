\chapter{Environnements tabulaires}

Il existe deux environnements basiques pour faire des tableaux en \LaTeX.


\section{Environnement tabular}

Du texte avant le tableau \ldots

\begin{tabular}{ccc} % lcr = left, centered, right
\hline
\multicolumn{2}{c}{Opérandes} & Résultat \\
$A$ & $B$ & $A+B$ \\
%\hline
\multirow{3}{*}{0} & 0 & 0 \\
 & 1 & 1 \\
 & 2 & 2 \\
\hline
\end{tabular}

Du texte après le tableau \ldots

\begin{figure}
\begin{center}
\begin{tabular}{ccc} % lcr = left, centered, right
\hline
\multicolumn{2}{c}{Opérandes} & Résultat \\
$A$ & $B$ & $A+B$ \\
%\hline
\multirow{3}{*}{0} & 0 & 0 \\
 & 1 & 1 \\
 & 2 & 2 \\
\hline
\end{tabular}
\end{center}
\caption{Les maths du XXIième siècle.}
\label{tab:maths}
\end{figure}

\section{L'environnement Table}


\begin{table}
\begin{center}
\begin{tabular}{ccc} % lcr = left, centered, right
\hline
\multicolumn{2}{c}{Opérandes} & Résultat \\
$A$ & $B$ & $A+B$ \\
%\hline
\multirow{3}{*}{0} & 0 & 0 \\
 & 1 & 1 \\
 & 2 & 2 \\
\hline
\end{tabular}
\end{center}
\caption{Les maths du XXIième siècle.}
\label{tab:maths}
\end{table}
