\documentclass[twoside, twocolumn, a4paper]{book}
% ECRIRE DU FRANCAIS (Ou autres)
\usepackage[utf8]{inputenc} 
\usepackage[T1]{fontenc}
\usepackage[francais]{babel}
\usepackage{lipsum}
\usepackage{blindtext}
\usepackage[Lenny]{fncychap}
\usepackage{amsmath}
\usepackage{SIunits} % Gestion des unités physiques
\blindmathtrue
\usepackage{hyperref} % Liens hyperref
\hypersetup{
    bookmarks=true,         % show bookmarks bar?
    unicode=false,          % non-Latin characters in Acrobat’s bookmarks
    pdftoolbar=true,        % show Acrobat’s toolbar?
    pdfmenubar=true,        % show Acrobat’s menu?
    pdffitwindow=false,     % window fit to page when opened
    pdfstartview={FitH},    % fits the width of the page to the window
    pdftitle={My title},    % title
    pdfauthor={Author},     % author
    pdfsubject={Subject},   % subject of the document
    pdfcreator={Creator},   % creator of the document
    pdfproducer={Producer}, % producer of the document
    pdfkeywords={keyword1, key2, key3}, % list of keywords
    pdfnewwindow=true,      % links in new PDF window
    colorlinks=true,       % false: boxed links; true: colored links
    linkcolor=green,          % color of internal links (change box color with linkbordercolor)
    citecolor=green,        % color of links to bibliography
    filecolor=magenta,      % color of file links
    urlcolor=cyan           % color of external links
}



% Un commentaire

\title{Mon premier document \LaTeX}

\author{Ludovic Charleux}
\date{Lundi !}


\begin{document} 
\maketitle
\tableofcontents
\frontmatter % Préambule

\chapter{Une introduction à \LaTeX}
\section{Taper du texte}
%\lipsum{1}
\blindtext[10]

\mainmatter % Corps du document
\chapter{Utilisation des mathématiques}
\section{Et faire des mathématiques}
\subsection{Les équations}

$$
a = \int_0^\infty \sin\alpha d\alpha u_i^j
$$

$$
\sum_i^j
$$

$$
M_{ij}
$$

$$
\underbrace{A = B}_{\mbox{Une équation}}
$$

J'ai fait des maths dans l'équation \ref{eq:einstein} (voir page \pageref{eq:einstein}).

\begin{equation}
E = mc^2
\label{eq:einstein}
\end{equation}

\begin{eqnarray}
A & = & 5 + 5 \nonumber \\
  & = & 10
\end{eqnarray}

\begin{align}
A & = & 5 +  5 & + 5 \\
  & = & 10     & + 5
\end{align}


$$
M = 
\begin{bmatrix}
1 & 2 & 3 \\
4 & 5 & 6 \\
7 & 8 & 9 
\end{bmatrix}
 = 0
$$


\subsection{Des maths en ligne}
On a une formule avec $\alpha$ dedans. 

$$
u = \cos \alpha
$$

\subsection{Les unités physiques avec siunits}

La mauvaise manière:
$$
P = 1MPa
$$

La bonne manière:

$$
A = \unit{4}\angstrom\per\second\squared 
$$

\chapter{Les packages indispensables}
\label{chap:packages}

\section{Hyperref}

Le package crée des liens internes au document et externes( web, \ldots):
\begin{itemize}
\item Un lien vers l'équation \ref{eq:einstein}, ou le chapitre \ref{chap:packages} sur les packages. Ça marche aussi avec les notes des bas de page\footnote{Une note de bas de page.}
\item Des liens externes \url{www.google.com}, ou \href{www.google.com}{un moteur de recherche.}
\end{itemize}

\begin{enumerate}
\item rtttt
  \begin{enumerate}
  \item Une sous liste,
  \item gggg
  \end{enumerate}
\item ttt
\end{enumerate}

\begin{description}
\item[Le lapin]: grandes oreilles,
\item[Le loup]: grandes dents,
\end{description}

\section{Geometry}



\backmatter
\appendix % annexes et bibliographie
\chapter{Une annexe}
\section{Une section}
\end{document}
